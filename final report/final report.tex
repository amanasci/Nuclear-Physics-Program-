\documentclass[a4paper]{article}

%% Language and font encodings
\usepackage[english]{babel}
\usepackage[utf8x]{inputenc}
\usepackage[T1]{fontenc}
\usepackage[compat=1.0.0]{tikz-feynman}
\usepackage{listings}


%% Sets page size and margins
\usepackage[a4paper,top=3cm,bottom=2cm,left=3cm,right=3cm,marginparwidth=1.75cm]{geometry}

%% Useful packages
\usepackage{amsfonts}
\usepackage{amsmath}
\usepackage{graphicx}
\graphicspath{ {./img/} }
\usepackage[colorlinks=true, allcolors=blue]{hyperref}
\usepackage{float}
\usepackage{enumerate}
\usepackage{subfig}

\title{REYES Nuclear Physics Mentoring Final Report}
\author{Aman Kumar}

\begin{document}
\maketitle
The past six weeks were incredible. I was introduced to amazing people and topics due this program. Thanks to Prof. Briceno and Prof. Jackura, along with the Old Dominion University.
During these amazing six weeks I learned a lot, starting from the Standard Model of Particle Physics to the Feynman Diagrams. I was already very interested in the field before joining the program, but 
after this program I am more aware about the field as well as the ongoing research in the field. 
\\ \\ \\
In this report I summarise my learning from the program. 


\section{Week 1 and Standard Model of Particle Physics}
The first week was all about the basics of the particle physics and introduction to the standard model. The key learning points from the week 1 were: 
\\
\begin{enumerate}
    \item Standard Model and intro to the fundamental forces.
    \item Intro to different classes of particles: Fermions and Bosons; along with their properties. 
    \item Intro to QCD and the laws. 
    \item Feynman Diagrams to explain particle interactions and Lattice QCD to get numerical solutions. 
\end{enumerate}

\section{Week 2 and Complex Numbers}
The second week was all about the complex numbers and their occurence in the physics. The key learning points from the week 2 are:
\\
\begin{enumerate}
    \item Intro to Scattering Theory. 
    \item Operations on Complex Numbers using Python
    \item Visualisation methods for complex numbers like Domain Colouring using Python.
\end{enumerate}


\section{Week 3 and Details of Scattering Theory}
Week 3 took us all to dive deeper into the world of Scattering Theory. The key learning points from week 3 are as follows: 
\\
\begin{enumerate}
    \item Features of Scattering Amplitude
    \item Intro to parameters like Phase shift and K matrix. 
    \item Plotting $\rho$ (Phase Space) for different particle interactions.
\end{enumerate}


\section{Week 4 and QCD Spectroscopy and Particle Data Group(PDG)}
Week 4 began with the introduction to the QCD Spectroscopy and moved on to the usage of data provided by PDG. The kry learning points from the week 4 are:
\\
\begin{enumerate}
    \item Intro to PDG and how to access data.
    \item Information on Decay Channels and Forces involved. 
    \item A brief introduction to Hadronic Resonances.
\end{enumerate}


\section{Week 5 and Feynman Diagrams}
This week was one of the most fun weeks. We dived deeper into the Feynman Diagrams and learnt following things: 
\\
\begin{enumerate}
    \item Feynman Diagrams and it's components. 
    \item Feynman Rules to convert Feynman Diagrams to equivalent mathematical equations.
    \item Intro to Symmetry Factor
\end{enumerate}

\section{Final Words}
Overall these 6 weeks taught me many things which I was unaware of initially. The assignments from both the proffesors were fun and engaging. They forced me to think deeper and review the things we learnt in lectures.
The thought provoking assignments helped me to explore deeper than what was being taught on my own. Sometimes to know more about something, sometimes to solve the assignments. 
\\ \\
The detail Information about each week is available in weekly reports. 
The tex files and code for the coding assignments are available on my github profile at: \href{https://github.com/amanasci/Nuclear-Physics-Program-}{https://github.com/amanasci/Nuclear-Physics-Program-}
\end{document}