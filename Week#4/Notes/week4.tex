\documentclass[a4paper]{article}



%% Language and font encodings
\usepackage[english]{babel}
\usepackage[utf8x]{inputenc}
\usepackage[T1]{fontenc}
\usepackage[compat=1.0.0]{tikz-feynman}
\usepackage{listings}


%% Sets page size and margins
\usepackage[a4paper,top=3cm,bottom=2cm,left=3cm,right=3cm,marginparwidth=1.75cm]{geometry}

%% Useful packages
\usepackage{amsfonts}
\usepackage{amsmath}
\usepackage{graphicx}
\graphicspath{ {./img/} }
\usepackage[colorlinks=true, allcolors=blue]{hyperref}
\usepackage{float}
\usepackage{enumerate}
\usepackage{subfig}

\title{REYES Nuclear Physics Mentoring Week 4}
\author{Aman Kumar}

\begin{document}
\maketitle
\section{Introduction}
This week learned about QCD spectroscopy and using data from Particle Data Group (PDG). The goal of QCD spectroscopy is to determine the states and 
properties of excited states as well as the intermediate states. When the excited state decays into multiparticle states, each decay channel 
has its own probability while conserving some fundamental properties like charge, energy, momentum, and spin. QCD spectroscopy helps us to study these channels. 
\\
Particle Data Group(PDG) is a reliable source for information about all the particles observed. 

\section{Mesonic Spectrum in PDG}
\begin{center}
    \begin{tabular}{c c c c c}
        Particle & Mass & Lifetime & Decay Channel & Force Involved \\
        $\pi^+ / \pi^-$ & 140MeV & $3 * 10^{-8}s$ & $\mu^+ \nu_\mu (~100\%)$  & weak \\
        $\pi^0$ & 135MeV & $9 * 10^{-17}s$ & $2 \gamma (99\%)$ & QED \\
        $\eta$ & 550MeV & $5 * 10^{-19}s$ & $2 \gamma(39\%) , 3 \pi^0 (32\%) , \pi^+ \pi^- \pi^0 (23\%)$ & $QED,QCD,QCD$ \\
    \end{tabular}
\end{center}

The General conclusion which we drew from the data above and more data is that, if a particle has very prominent decay channels 
through very strong forces like QCD, the lifetime of the particle is usually shorter. So, the Standard Model without Weak forces or QED is sometimes
used as a model to further syudy QCD decays since it is the most difficult one to study, due to the very short lifetimes of the particles.
\\
Also if a particle has enough energy or mass to go to simpler forms, it'll decay. 
\\
The simplified standard model has no decay channels due to weak forces or QED.
\\
The heavier the particle is, the larger the number of channels through which it can decays as there are more intermediate states possible to reach.
An example is, $f_0(980)$ with a mass of around 990MeV. Due to its heavy mass, it can decay into two kaons or two pions. Both having energy lower than it.

\section{More info}
Further more we learnt about Hadronic resonances and different resonant reactions are studied at different research labs along with the represntation of the mass matrix 
used in scattering amplitude analysis.

\end{document}